\documentclass[10pt,a4paper, margin=1in]{article}
\usepackage{fullpage}
\usepackage{amsfonts, amsmath, pifont}
\usepackage{amsthm}
\usepackage{graphicx}
\usepackage[utf8]{inputenc}

\usepackage{fullpage}
\usepackage{amsfonts, amsmath, pifont}
\usepackage{amsthm}
\usepackage{graphicx}
\usepackage{float}

\usepackage{tkz-euclide}
\usepackage{tikz}
\usepackage{pgfplots}
\pgfplotsset{compat=1.13}
\usepackage{geometry}

 \geometry{
 a4paper,
 total={210mm,297mm},
 left=10mm,
 right=10mm,
 top=10mm,
 bottom=10mm,
 }
 % Write both of your names here. Fill exxxxxxx with your ceng mail address.
 \author{
  Özalp, Zeynep\\
  \texttt{e2237691@ceng.metu.edu.tr}
  \and
  Yıldırım, Hilmi Cihan\\
  \texttt{e2237949@ceng.metu.edu.tr}
}
\title{CENG 384 - Signals and Systems for Computer Engineers \\
Spring 2020 \\
Written Assignment 2}
\begin{filecontents}{q3.dat}
 n   xn
 -2  2
 -1  4
 0   1
 1   2  
 2   0
 3   0
 4   0
 5	 0 

\end{filecontents}
\begin{document}
\maketitle



\noindent\rule{19cm}{1.2pt}

\begin{enumerate}

\item %write the solution of q1
    \begin{enumerate}
    % Write your solutions in the following items.
    \item %write the solution of q1a\\
    
   	\textbf{memory} \\
   	y[0]=y[-1]+y[-2]+.. so we need memory because the output y[5] depends upon the past value of x[-1] so we need memory\\
   	\textbf{stability} \\
   	we have to find bounded inputs give or give not bounded outputs so \\
   	it is unstable because if B<x[n]<B y[n] goes infinity because there is infity sum of B's 
   	\\
   		\textbf{Linearity} \\
   		Let us apply superposition property to determine linearity \\
   		y1[n] =x1[n-1]+x1[n-2]+...\\
   		y2[n] =x2[n-1]+x2[n-2]+... \\
	now let us consider a third input x3[n] such that it is linear combination of x1[n] and x2[n]  \\   	
		x3[n]=ax1[n]+bx2[n] \\ 
		therefore ,the output y3[n] is given as \\ 
		y3[n]=x3[n-1]+x3[n-2]+... \\
		y3[n] = ax1[n-1]+bx2[n-1]+ax1[n-2]+bx2[n-2]... \\ 
		y3[n] = a( x1[n-1]+x1[n-2]+...)+b(x2[n-1]+x2[n-2]+...) \\
		y3[n] = y1[n]+y2[n] \\ 
		From the above expressions we conclude system is linear we can easily see that  both additivity and homogeneity proporteis hold. \\
		\textbf{Invertibilty} \\
	A system is invertible if distinct inputs lead distinct outputs \\
	we can easily see that system performs summation of inputs .For different inputs  the outputs are different. so system is invertible\\ 
	\textbf{Time -Invariance} \\
	To check that this system is time invariant, we must determine whether the timeinvariance property holds for any input and any time shift n0 Thus, let x1[n-n0] be an arbitrary input to this system, and let \\
	y1[n]= x1[n-1]+x1[n-2]+.. \\ \\ 
	x2[n]=x1[n-n0] \\
	y2[n]=x2[n-1]+x2[n-2]+... =x1[n-n0-1]+x1[n-n0-2]+... \\
	so \\
	y2[n]=y1[n-n0] therefore the system is time invariant. \\ 
	\textbf{Conculusion} \\
	\textbf{the system is has memory, unstable, linear, invertible and time invariant}
	
	
	
	
    \item %write the solution of q1b
    	\textbf{memory} \\
    	A system is said to be memory less if its output for each value of the independent variable
at a given time is dependent only on the input at that same time.\\

   	y(1)=y(5) so system \textbf{ need memory } because is not dependent only on the input at that same time. \\
   		\textbf{stability} \\
   	let us consider $|x(t)|< \infty$ \\
   	$|y(t)|=|ty(2t+3)|$ \\
   	$|y(t)| \leq |t||y(2t+3)|$ \\
   	even if $|x(t)|< \infty$ the magnitude of the output depends upon the variable n,which states that the output is \textbf{unstable }\\
   	\textbf{Linearity} \\
   	y1(t) = t x1(2t+3) \\ 
   	y2(t) = t x2(2t+3) \\
   	\\
   	x3(t) = ax1(t)+bx2(t) \\
   	 \\
   	 y3(t) = tx3(2t+1)
   	 \\
   	 y3(t) = atx1(2t+3)+btx2(2t+3) = ay1(t)+by2(t) so this system is linear\\
   	 \textbf{Invertibilty} \\
	A system is invertible if distinct inputs lead distinct outputs \\
	this system is not invertible because for example if x(1)=5 and x(5)=1 then y(1)=1*5 and y(5)=5*1 so system give same output for different inputs system is \textbf{ not invetible}.\\
	
   	\textbf{Time -Invariance} \\
	To check that this system is time invariant, we must determine whether the timeinvariance property holds for any input and any time shift n0 Thus, let x1[n-n0] be an arbitrary input to this system, and let \\
	x(t-t0) $\rightarrow$ y(n)=nx(n-n0) but y(n-n0)=(n-n0)x(n-n0)tehy are not equal so system is \textbf{ not time invariant }
	\\
	
	
   	
    
    \end{enumerate}

\item %write the solution of q2
    \begin{enumerate}
    % Write your solutions in the following items.
    \item %write the solution of q2a
    $y(t)=\int_{-\infty}^{\infty}(x(\tau)-5y(\tau))d\tau$\\
    Differentiate both sides. \\
    $\dfrac{dy(t)}{dt}=x(t)-5y(t)$\\
    $\dfrac{dy(t)}{dt}+5y(t)=x(t)$\\
    $y'(t)+5y(t)=x(t)$
    \item %write the solution of q2b
    $ y'(t)+5y(t)=e^{-t}+e^{-3t} $\\
    First, homogenous solution:\\
    $\lambda+5=0 \Rightarrow \lambda = -5$\\
    $ y_H(t)=Ke^{-5t} $\\
    For particular solution, since the system is linear, find particular solution to $x_1(t)=e^{-t}$ and $x_2(t)=e^{-3t}$ separately. Then combine.\\
    1) For $x_1(t)=e^{-t}$, assume $y_{P1}=A_1e^{-t}$\\
    $-A_1e^{-t}+5A_1e^{-t}=e^{-t} \Rightarrow A_1=1/4$\\
    $y_{P1}=(1/4)e^{-t}$\\
    $y_{G1}=Ke^{-5t}+(1/4)e^{-t}$\\
    2) For $x_2(t)=e^{-3t}$, assume $y_{P2}=A_2e^{-3t}$\\
    $-3A_2e^{-3t}+5A_2e^{-3t}=A_2e^{-3t} \Rightarrow A_2=1/2$\\
    $y_{P2}=(1/2)e^{-3t}$\\
    $y_{G2}=Ke^{-5t}+(1/2)e^{-3t}$\\
    So, general solution is $y(t)=2Ke^{-5t}+(1/4)e^{-t}+(1/2)e^{-3t}$\\
    $y(0)=2K+3/4=0 \Rightarrow K=-3/8$\\
    $y(t)=-(3/4)e^{-5t}+(1/4)e^{-t}+(1/2)e^{-3t}$
    \end{enumerate}

\item %write the solution of q3     
    \begin{enumerate}
    % Write your solutions in the following items.
    \item %write the solution of q3a
    y[n] = x[n] * h[n]= $\sum\limits_{k= -\infty }^\infty x[k] h[n-k] $ \\
    
    x[n] = $\sum\limits_{k= -\infty }^\infty x[k] \delta [n-k] $ \\ according to above equations from book \\
    y[n] = $\sum\limits_{k= -\infty }^\infty x[k]   \delta [n-1-k] $+  2$\sum\limits_{k= -\infty }^\infty x[k]   \delta [n+1-k] $
    \\
    x[n-1]=$\sum\limits_{k= -\infty }^\infty x[k] \delta [n-1-k] $ \\
    x[n+1]=$\sum\limits_{k= -\infty }^\infty x[k] \delta [n+1-k] $ \\
    so \\
    y[n] = x[n-1]+2x[n+1] \\
    y[n] =  $2 \delta [n-1]+ \delta [n]+2(2 \delta [n+1]+\delta [n+2])       $ \\
    \begin{figure} [h!]
    \centering
    \begin{tikzpicture}[scale=1.0] 
      \begin{axis}[
          axis lines=middle,
          xlabel={$n$},
          ylabel={$\boldsymbol{y[n]}$},
          xtick={ -2, -1, 0,  1,2},
          ytick={2,4,1,2},
          ymin=-4, ymax=6,
          xmin=-5, xmax=5,
          every axis x label/.style={at={(ticklabel* cs:1.05)}, anchor=west,},
          every axis y label/.style={at={(ticklabel* cs:1.05)}, anchor=south,},
          grid,
        ]
        \addplot [ycomb, black, thick, mark=*] table [x={n}, y={xn}] {q3.dat};
      \end{axis}
    \end{tikzpicture}
    \caption{$n$ vs. $x[n]$.}
    
\end{figure}
   
    
    \item %write the solution of q3b
 $ \dfrac{d}{dt} u(t-t0) = \delta (t-t0)$ from book  therefore \\ 
    	 $ \dfrac{d}{dt} x(t) =  \delta (t-1)+ \delta (t+1) $ \\
    	 % Write your solutions in the following items.
    
   
    y(t) = $ (\delta (t-1)+ \delta (t+1))h(t) $ \\
    from distribution property of convolution \\
     y(t) = $ \delta (t-1)*h(t) + \delta (t+1)*h(t)) $ \\
    from book x(t)$\delta (t-t0) = x(t-t0)$ therefore \\
    y(t) = h(t-1)+h(t+1) \\
    y(t) = $e^{-t+1}sin(t-1)u(t-1)+e^{-t-1}sin(t+1)u(t+1) $ \\
    y(t)=0 for t$<$ -1 \\
    y(t) = $e^{-t-1}sin(t+1)u(t+1) $ for -1$\leq$t$<$1 \\
    y(t) = $e^{-t+1}sin(t-1)u(t-1)+e^{-t-1}sin(t+1)u(t+1) $ for 1$\leq$t
     
    	 
    \end{enumerate}

\item %write the solution of q4
    \begin{enumerate}
    % Write your solutions in the following items.
    \item %write the solution of q4a
    The product of $x(\tau)\ and\ h(t-\tau)$ is non zero for only $0<\tau<t$ \\
    For $t\geq0$:
    $\int_{0}^{t}x(\tau)h(t-\tau)d\tau=\int_{0}^{t}e^{-\tau}e^{-2t}e^{2\tau}d\tau=e^{-2t}\int_{0}^{t}e^{\tau}d\tau=e^{-2t}(e^t-1)=(e^{-t}-e^{-2t})$\\
    For $t<0$: y(t)=0\\
    Thus, $y(t)=(e^{-t}-e^{-2t})u(t)$
    \item %write the solution of q4b
    The product of $x(\tau)\ and\ h(t-\tau)$ is nonzero for only $0\leq\tau\leq1$. Also,  because $x(\tau)=1$ for $0\leq\tau\leq1$, we do not need to write it in the product.\\
    $\int_{0}^{1}x(\tau)h(t-\tau)d\tau=y(t)=\int_{0}^{1}e^{3t-3\tau}d\tau=e^{3t}\int_{0}^{1}e^{-3\tau}d\tau=e^{3t}\dfrac{e^{-3}-1}{-3}=\dfrac{-e^{3t-3}+e^{3t}}{3}$\\
    Thus, \begin{equation}
   y(t)=% 
   \begin{cases}
     \dfrac{-e^{3t-3}+e^{3t}}{3} & 0\leq t\leq1 \\
     0 &\text{otherwise}
   \end{cases}
\end{equation}
    \end{enumerate}

\item %write the solution of q5
    \begin{enumerate}
    % Write your solutions in the following items.
    \item %write the solution of q5a
    \item %write the solution of q5b
    we assume y(t)=$Ae^{rt} $ \\ 
    so \\
    $y'(t)= Ar e^{rt}$ \\
    $y''(t)= Ar^2 e^{rt}$ \\
    $y'''(t)= Ar^3 e^{rt}$ \\
    so \\
    $r^3 e^{rt} -3(r^2 e^{rt})+4(r e^{rt})-2(e^{rt})$ \\
    $Ae^{rt} (r^3-3r^2+4r-2)=0$ $e^{rt} $cannot be zero so $ (r^3-3r^2+4r-2)=0$ need to be true \\
    so r1=1 \\
    r2=1-j \\
    r3=1+j \\
    y(t)=c1 $e^{t} $+$e^{t}$ (c2sin(t)+c3cos(t))  \\
    y(0)=c1+c3=3 \\
    y'(t)=c1 $e^{t} $+$e^{t}$ (c2sin(t)+c3cos(t))+$e^{t}$ (c2cos(t)-c3sin(t)) \\
    y'(0)=c1+c2+c3=1 \\
    y''(t)=c1 $e^{t} $+$e^{t}$ (c2sin(t)+c3cos(t))+$e^{t}$ (c2cos(t)-c3sin(t)) +$e^{t}$ (c2cos(t)+c3 -sin(t)) +$e^{t}$ (c2-sin(t)-c3cos(t)) \\
    y''(0)=c1+2c2=2 \\
    c1=6 \\
    c2=-2 \\
    c3=-3 \\
     
    
    \end{enumerate}


\item %write the solution of q6
    \begin{enumerate}
    % Write your solutions in the following items.
    \item %write the solution of q6a
    $ w[n]-\dfrac{1}{2}w[n-1]=x[n] $\\
    Take the Fourier Transform of both sides to be in frequency domain.\\
    $ W(e^{jw})-\dfrac{1}{2}e^{-jw}W(e^{jw})=X(e^{jw}) $\\
    $ W(e^{jw})(1-\dfrac{1}{2}e^{-jw})=X(e^{jw}) $\\
    $\dfrac{W(e^{jw})}{X(e^{jw})}=H_0(e^{jw})=\dfrac{1}{1-\dfrac{1}{2}e^{-jw}}$\\
    Now, take the reverse fourier to be in time domain.\\
    $ h_0[n]=(\dfrac{1}{2})^nu[n] $
    \item %write the solution of q6b
    $ h_0[k]h_0[n-k] $ is positive only for $0\leq k \leq n$. \\
    $ h_0[n] \ast h_0[n]=\Sigma_0^n h_0[k]h_0[n-k]=\Sigma_0^n(\dfrac{1}{2})^n $\\
    $ h[n]=n(\dfrac{1}{2})^nu[n]$
    \item %write the solution of q6c
    $ X(e^{jw})H_0(e^{jw})H_0(e^{jw})= W(e^{jw})$\\
    $ \dfrac{Y(e^{jw})}{X(e^{jw})} = (H_0(e^{jw}))^2 = \dfrac{1}{1-e^{-jw}-\frac{1}{4}e^{-2jw}} $\\
    $ Y(e^{jw})-e^{-jw}Y(e^{jw})-\frac{1}{4}e^{-2jw}Y(e^{jw}) = X(e^{jw}) $\\
    Take the inverse fourier to be in time domain.\\
    $ y[n]-y[n-1]-\frac{1}{4}y[n-2]=x[n] $
    \end{enumerate}

\end{enumerate}
\end{document}

